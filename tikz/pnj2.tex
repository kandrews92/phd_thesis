\documentclass[tikz]{standalone}

\usepackage{tikz}
\usetikzlibrary{patterns, arrows, calc, decorations.pathmorphing}

% Modified \textcircled marco
\renewcommand*\textcircled[1]{\tikz[baseline=(char.base)]{
	\node [shape=circle, draw, inner sep=1pt] (char) {#1};}}

\begin{document}

\begin{tikzpicture}
	% variable for pn-junction diagram:
	% all parameters are in tikz scale
	% p-side of the junction is here on the right

		\def\V{1.5}	% junction polarization (0 = flat band)
		\def\EG{3}	% band gap of semiconductor
		\def\EF{1.5}	% vertical fermi level position
		\def\EFn{3.3}	% psuedo fermi level for electrons
		\def\EFp{1.8}	% psuedo fermi level for holes
		\def\DZCE{4}	% start position on the left for space charge region (SCR)
		\def\LZCE{2}	% SCR width
		\def\PN{10}	% total length of the junction
	
	% calculations
	\pgfmathsetmacro\EC{\EG+\V}; % conduction band hegiht (w/o polarization)
	\pgfmathsetmacro\FZCE{\DZCE+\LZCE}; % SCR end position

	% valence and conduction band drawing:
	\draw (0,0) node [left]{$E_V$} -- (\DZCE,0)
		to[out=0, in=180, looseness=0.75] (\FZCE,\V) -- (\PN,\V); % EV
	\draw (0,\EG) node [left] {$E_C$} -- (\DZCE,\EG)
	    to[out=0, in=180, looseness=0.75] (\FZCE,\EC) -- (\PN,\EC); % EC

  	% fermi level drawing (if needed):		
    \draw [dashed] (0,\EF) node [left]{$E_F$} -- ({\PN-0.5},\EF);

  	% quasi fermi levels drawing (if needed) :	
  	\draw [dashed] (0,\EFn)  -- (\PN,\EFn)
    	node [right] {$E_{Fn}$}; % EFn for electron
  	\draw [dashed] (0,\EFp)  -- ({\PN},\EFp)
	    node [right] {$E_{Fp}$}; % EFp for holes

  	% electric field in SCR drawing :
  	%\draw [->] (\DZCE, {\V+\EG+1}) --
    %	node [above] {$\vec{E}_{ZCE}$} (\FZCE, {\V+\EG+1}) ; % E vector

  	% excess carriers
  	\foreach \x in {1,2,...,7}
    	\draw ({\FZCE+1+\x/3},{\EC+0.2}) node {$\bullet$}; % p side : electrons
  	\foreach \x in {1,2,...,7}
    	\draw ({1+\x/3},{-0.2}) node {$\circ$}; % n side : holes

  	% photon injection and carrier generation
  	% p side : carrier generation:
  	%\draw [->, dashed] ({\FZCE+3}, \V) --
    %	node [right] {\textcircled{1}}({\FZCE+3}, \EC);
  	% the textcircled{number} option is used in several places
  	% to describe the physical mechanisms.
  	% It can be safely removed if not needed
  	% photon wave injection in the bandgap on p-side :
  	
    %\draw [decorate, decoration={snake}, ->] ({\PN+1},{\EC+1}) --
    %	node [below,sloped]{$h\nu$} ({\FZCE+3}, \EG); 
		
  	% excess carriers diffusion, with diffusion length :
  	% electrons on p side :
    \draw [->] ({\FZCE+1},{\EC+0.2}) -- node [above] {$L_{Dn}$} ({\FZCE},{\EC+0.2})
      node [left] {$\bullet$};
    \draw [->] ({\FZCE-0.3},{\EC+0.2}) to [out=200, in=0, looseness=0.75] ({\DZCE},{\EG+0.2})
      node [left] {$\bullet$};

  	% holes on n side :
  	\draw [->] ({1.2+7/3},{-0.2}) -- node [below] {$L_{Dp}$} ({\DZCE},{-0.2})
    	node [right]{$\circ$} ;
  	\draw [->] ({\DZCE+0.3},-0.2) to [out=20, in=180, looseness=0.75]
    	({\FZCE},{\V-0.2}) node [right]{$\circ$};

\end{tikzpicture}

\end{document}
