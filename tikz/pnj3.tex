%\documentclass[a4paper]{article}
\documentclass[tikz]{standalone}

\usepackage{tikz}
\usetikzlibrary{patterns, arrows, calc, decorations.pathmorphing}

\begin{document}

\begin{tikzpicture}
	% variable for pn-junction diagram:
	% all parameters are in tikz scale
	% p-side of the junction is here on the right

    \def\EFm{2}         % metal fermi level position
    \def\EF{2.75}       % semiconductor fermi level 
    \def\MetalEnd{3}    % total length of the metal
    \def\SemiEnd{8}     % total length of the semiconductor
    \def\V{1.0}         % base level
    \def\EG{1.8}        % energy gap of semiconductor

  % calculations
  \pgfmathsetmacro\totalLen{\MetalEnd+\SemiEnd};  % total length of the image
  \pgfmathsetmacro\EC{\EF+\V};                    % conduction band height
  \pgfmathsetmacro\EV{\EF-\V};

  % draw metal contact fermi level
  \draw [dashed] (\V,\EFm) node [left]{$E_{Fm}$} -- (\MetalEnd, \EFm); % EFm
  % draw hashed region of metal contact
  \draw[pattern = north west lines, draw=none] (\MetalEnd,2) rectangle (2,1.5);
  \draw[draw=none] (1, 1.5) rectangle (0,0);

  % draw semiconductor fermi level
  \draw [dashed] (\MetalEnd,\EF) -- (\totalLen,\EF) node [right]{$E_F$};

  % draw valence band
  \draw (\MetalEnd,\EV) -- (\totalLen,\EV) node [right]{$E_V$};
  % draw conduction band
  \draw (\MetalEnd,\EC) -- (\totalLen,\EC) node [right]{$E_C$};

\end{tikzpicture}

\end{document}
